\documentclass[../report.tex]{subfiles}
\graphicspath{{\subfix{../images/}}}

\begin{document}
\section{Introduction}
\textbf{Routing in wireless ad hoc networks} presents significant challenges due to the dynamic nature of nodes and the absence of centralized control. This study investigates the performance of the \textbf{Link Reversal Algorithm} (LRA\cite{gafni1981distributed}) in a simulated environment using the \textbf{ns-3 network simulator}\cite{ns3}. The simulation parameters include varying node densities and network area sizes, with a focus on packet loss rates. The results provide insights into the \textbf{scalability} and robustness of LRA in different network conditions. The context of this work is \textbf{disaster recovery scenarios} where traditional communication infrastructure, such as cellular antennas, is unavailable, and a peer-to-peer (\textbf{P2P}) network is used for communication. In such scenarios, maintaining \textbf{network connectivity} and ensuring \textbf{reliable data transmission} are critical, especially when nodes are highly mobile or when environmental factors cause \textbf{frequent topology changes}.\footnote{
The source code for this study is available on GitHub: \url{https://github.com/GabrieleMessina/p2p_wireless_networks_course}
}
\end{document}