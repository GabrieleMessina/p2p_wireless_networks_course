\documentclass[../report.tex]{subfiles}
\graphicspath{{\subfix{../images/}}}

\begin{document}
\section{Work}

\subsection{Simulation Setup}
The simulation is designed to work with input parameters to assess the performance of the LRA protocol. The key simulation parameters include a variable \textbf{number of nodes}, deployed within a square \textbf{area} of chosen dimensions. Nodes are positioned uniformly at random to ensure a realistic network topology.

To introduce mobility, a \textbf{random walk model} is employed, enabling nodes to move freely within the designated area. The \textbf{speed} attribute of the walk model follows a Pareto distribution to simulate a context in which a few nodes move with a vehicle while the majority move by foot. This setup reflects real-world \textbf{disaster recovery} scenarios where emergency responders or supply vehicles travel faster than individuals navigating on foot.
At the beginning of the simulation, a \textbf{sink node} is designated, serving as the central recipient of network messages.

Each node randomly transmits a fixed number of \textbf{echo messages} to the sink exploiting NS3 \textbf{UdpEchoServerHelper} and \textbf{UdpEchoServerClient} fuctionalities and emulating data communication within the network. The total simulation duration is set from a parameter provided by the user.

The \textbf{LRA routing protocol} governs routing decisions, utilizing \textbf{link reversal mechanisms} to dynamically adapt to changes in node positions. The success or failure of packet transmissions is monitored, with \textbf{packet loss} recorded as a critical metric.

Here is a comprehensive list of the simulation parameters:
\begin{itemize}
    \item Number of Nodes
    \item Number of Packets
    \item Area Side Length
    \item Simulation Time
\end{itemize}

\begin{figure}[H] % Prevents page breaks
\begin{lstlisting}[language=C++, caption={Simulation Node Initialization},captionpos=b]
void LraExample::SetupNodes(int nNodes, double squareSize, double totalTime){
    ns3::NodeContainer nodes(nNodes);
    MobilityHelper mobility;
    mobility.SetPositionAllocator(
      "ns3::RandomRectanglePositionAllocator",
      "X", StringValue("ns3::UniformRandomVariable[Min=0|Max="+squareSize+"]"),
      "Y", StringValue("ns3::UniformRandomVariable[Min=0|Max="+squareSize+"]")
    );
    
    mobility.SetMobilityModel(
      "ns3::RandomWalk2dMobilityModel",
      "Mode", StringValue("Time"),
      "Time", StringValue(std::to_string(totalTime) + "s"),
      // Pareto: mean = 12, median = 8, min = 6, max = 80
      "Speed", StringValue("ns3::ParetoRandomVariable[Bound=80|Scale=6|Shape=2]"),
      "Bounds", RectangleValue(Rectangle(0, squareSize, 0, squareSize))
    );
    mobility.Install(nodes);
}
\end{lstlisting}
\end{figure}

\begin{figure}[H] % Prevents page breaks
\begin{lstlisting}[language=C++, caption={Application Simulation Initialization},captionpos=b]
void LraExample::SetupNetwork(){
    for (uint32_t i = 0; i < nodes.GetN(); ++i)
    {
        Ptr<Node> node = nodes.Get(i);
        Ptr<LraRoutingProtocol> lraRouting = node->GetObject<LraRoutingProtocol>();
        lraRouting->InitializeNode(m_sinkAddress);
    }

    int echoPort = 9;
    UdpEchoServerHelper echoServer(echoPort);
    ApplicationContainer echoServerApps = echoServer.Install(nodes);
    echoServerApps.Start(Seconds(0));
    echoServerApps.Stop(Seconds(totalTime));

    UdpEchoClientHelper echoClient(Address(m_sinkAddress), echoPort);
    echoClient.SetAttribute("MaxPackets", UintegerValue(n_packets));
    echoClient.SetAttribute("Interval", TimeValue(Seconds(1.0)));
    echoClient.SetAttribute("PacketSize", UintegerValue(32));

    for (uint32_t i = 0; i < nodes.GetN() - 1; ++i)
    {
        auto app = echoClient.Install(nodes.Get(i));
        int randDelay = rand() % 1000;
        app.Start(Seconds(startDelay + randDelay));
        app.Stop(Seconds(totalTime));
    }

    // Trace package send and receive to calculate loss
    Config::Connect(
        "/NodeList/*/ApplicationList/*/$ns3::UdpEchoServer/RxWithAddresses",
        MakeCallback(&LraExample::LogMessageResponse, this)
    );
    Config::Connect(
        "/NodeList/*/ApplicationList/*/$ns3::UdpEchoClient/TxWithAddresses",
        MakeCallback(&LraExample::LogMessageSend, this)
    );
}
\end{lstlisting}
\end{figure}


\subsection{Network Configuration}
The network is established using an \textbf{IEEE 802.11ac WiFi standard} with an \textbf{ad hoc} configuration. The wireless network is configured as follows:

\begin{itemize}
    \item The MAC layer is set to \textbf{AdhocWifiMac}, ensuring decentralized network operation.
    \item The physical layer uses the \textbf{YansWifiPhy} model, with parameters set for carrier sense threshold and sensitivity at \textbf{zero}, optimizing the detection of nearby transmissions.
    \item The wireless channel follows the \textbf{Default YansWifiChannelHelper} configuration.
    \item The remote station manager utilizes \textbf{IdealWifiManager}, optimizing the transmission rate selection.
\end{itemize}

The Internet stack is deployed as follows:

\begin{itemize}
    \item Nodes are assigned IP addresses within the \textbf{10.0.0.0/8} subnet.
    \item The sink node is designated as the last node in the list and assigned the highest IP address.
\end{itemize}

\begin{figure}[H] % Prevents page breaks
\begin{lstlisting}[language=C++, caption={Simulation Network Setup},captionpos=b]
void LraExample::SetupNetwork(){
    WifiMacHelper wifiMac("ns3::AdhocWifiMac");
    
    YansWifiPhyHelper wifiPhy;
    wifiPhy.Set("CcaEdThreshold", DoubleValue (0));
    wifiPhy.Set("CcaSensitivity", DoubleValue (0));
    
    auto wifiChannel = YansWifiChannelHelper::Default();
    wifiPhy.SetChannel(wifiChannel.Create());
    
    WifiHelper wifi;
    wifi.SetStandard(WIFI_STANDARD_80211ac);
    wifi.SetRemoteStationManager("ns3::IdealWifiManager");
    wifi.Install(wifiPhy, wifiMac, nodes);

    stack.SetRoutingHelper(LraHelper());
    stack.Install(nodes);

    Ipv4AddressHelper address;
    address.SetBase("10.0.0.0", "255.0.0.0");
    ipv4Interfaces = address.Assign(netDevices);
    m_sinkAddress = ipv4Interfaces.GetAddress(nodes.GetN() - 1);
}
\end{lstlisting}
\end{figure}


\subsection{Routing protocol Implementation}
To implement the proposed protocol, a new \textbf{NS3} component, \textbf{LraRoutingProtocol}, has been developed. This component extends the \textbf{Ipv4RoutingProtocol} class provided by the NS3 framework.

The routing table is maintained using a \textbf{map data structure}, which establishes a correlation between an \textbf{IPv4 address} and its corresponding \textbf{LRA Link Status}.

Upon attachment of the routing protocol to a node, the node initiates a \textbf{broadcast service packet} known as the \textbf{Hello Message}. It subsequently awaits responses from neighboring nodes, which reply with a \textbf{Hello Response Message}. This exchange enables the node to \textbf{discover its neighbors} and construct its \textbf{routing table} accordingly.

When a \textbf{non-service packet} is received, the node determines whether it is the intended destination. If the node is the \textbf{destination}, the protocol terminates successfully. If the node is \textbf{not} the destination or if it is \textbf{creating and forwarding a new packet} towards the sink, it searches for the \textbf{next hop} within its routing table. 

\begin{itemize}
    \item If a \textbf{valid} next \textbf{hop} is found, the packet is forwarded to that hop along with an \textbf{Ack Request Message}, and a \textbf{timer} is started. If no \textbf{Ack Response Message} is received before the timer \textbf{expires}, the link toward that hop is marked as \textbf{"Ingoing"}.
\end{itemize}

\begin{itemize}
    \item If \textbf{no} valid next \textbf{hop} is found, the \textbf{Link Reversal} mechanism is triggered, and the search process is repeated. If, after multiple attempts, no next hop is identified, the node \textbf{disables itself}, as it is determined to be part of a subgraph that is \textbf{disconnected from the sink}.
\end{itemize}

When a \textbf{service packet} is received, the node processes it based on the \textbf{packet type}, as follows:

\begin{itemize}
    \item \textbf{Hello Message}: The receiving node updates its routing table by adding the sender with a \textbf{link type of "Outgoing"} if the sender's \textbf{IP address} is \textbf{greater} than that of the recipient; otherwise, the link is classified as \textbf{"Ingoing"}.
    \item \textbf{Hello Response Message}: The behavior is identical to that of the \textbf{Hello Message}, where the link type is determined based on the comparison of IP addresses.
    \item \textbf{Ack Request Message}: The receiving node responds by transmitting an \textbf{Ack Response Message}.
    \item \textbf{Ack Response Message}: Upon reception, the node halts the execution of the \textbf{DisableLinkTimer} associated with the sender.
    \item \textbf{Link Reversal Message}: The receiving node updates its routing table by designating the \textbf{link towards the sender} as \textbf{"Ingoing"}. Additionally, it evaluates whether a \textbf{Link Reversal} is necessary.
\end{itemize}


\begin{figure}[H] % Prevents page breaks
\begin{lstlisting}[language=C++, caption={Link Reversal Method},captionpos=b]
void LraRoutingProtocol::LinkReversal()
{
    // Base case
    if (m_nodeAddress == m_sink)
    {
        return;
    }

    // Actual inversion
    for (auto neighbor : m_neighbors)
    {
        m_linkStatus[neighbor] = 1; //Outgoing
    }
}
\end{lstlisting}
\end{figure}

\begin{figure}[H] % Prevents page breaks
\begin{lstlisting}[language=C++, caption={Next Hop Find Method},captionpos=b]
Ipv4Address LraRoutingProtocol::GetNextHop(){
    auto nextHop = GetFirstOutgoingLink();
    if(nextHop != m_broadcastAddress){
        return nextHop;
    }
    else{
        if(nextHop == m_broadcastAddress && m_neighbors.size() > 0){
            LinkReversal();
            auto nextHop = GetFirstOutgoingLink();
            // Notify all nodes that you are now in active state.
            SendReversalMessage(m_broadcastAddress);
            return nextHop;
        }
    }

    return m_broadcastAddress; // fallback address
}
\end{lstlisting}
\end{figure}


\begin{figure}[H] % Prevents page breaks
\begin{lstlisting}[language=C++, caption={Service Packet Received Handling Method},captionpos=b]
void LraRoutingProtocol::RecvLraServicePacket(Ptr<Packet> packet, Ipv4Address origin)
{
    std::string payload = GetPacketPayload(packet);

    // Ack Request Message received
    if (payload == LraRoutingProtocol::LRA_ACK_SEND_MESSAGE)
    {
        if (m_linkStatus[origin] == 1) // Cycle detected
        {
            DisableLinkTo(origin);
            return;
        }
        SendAckResponseMessage(origin);
    }
    // Ack Response received
    else if (payload == LraRoutingProtocol::LRA_ACK_RECV_MESSAGE)
    {
        m_disableLinkToEvent[origin].Cancel(); // link active, avoid disabling it
        m_disableLinkToEvent.erase(origin);
    }
    // Hello Message received
    else if (payload == LraRoutingProtocol::LRA_HELLO_SEND_MESSAGE)
    {
        if (m_nodeAddress < origin)
            EnableLinkTo(origin);
        else
            DisableLinkTo(origin, true);

        SendHelloResponseMessage(origin);
    }
    // Hello Message Response received
    else if (payload == LraRoutingProtocol::LRA_HELLO_RECV_MESSAGE)
    {
        if (m_nodeAddress < origin)
            EnableLinkTo(origin);
        else
            DisableLinkTo(origin, true);
    }
    // Link Reversal Message received
    else if (payload == LraRoutingProtocol::LRA_REVERSAL_SEND_MESSAGE)
    {
        DisableLinkTo(origin);
    }
}
\end{lstlisting}
\end{figure}

\end{document}