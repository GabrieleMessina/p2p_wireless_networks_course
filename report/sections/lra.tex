\documentclass[../report.tex]{subfiles}
\graphicspath{{\subfix{../images/}}}

\begin{document}
\section{Link Reversal Algorithm}
The \textbf{Link Reversal Algorithm} (\textbf{LRA}\cite{gafni1981distributed}) is a distributed routing protocol mainly used in dynamic and mobile network environments. It is particularly suited for ad hoc and wireless networks where network topology changes frequently. The algorithm operates by reversing the direction of links between nodes to restore lost routes, ensuring uninterrupted connectivity even in the presence of frequent topology changes.

LRA relies on the concept of maintaining a \textbf{Directed Acyclic Graph} (\textbf{DAG}) rooted at a designated sink node. In a DAG, each node has links pointing towards one or more downstream nodes, with the sink node serving as the ultimate destination. When a link failure occurs, LRA dynamically reverses the direction of affected links, initiating a localized response to restore the DAG structure. This process avoids the need for global routing updates, reducing control overhead and improving scalability.

The operation of LRA can be divided into four main stages. First, during initialization, the network establishes a DAG, ensuring that all nodes can route packets to the sink. Second, nodes consistently observe their connections for faults, particularly the inability to reach the sink, using mechanisms such as acknowledgment timeouts or signal strength analysis. Third, upon detecting a failure, affected nodes reverse their outgoing links, which may propagate asynchronous reversals to neighbouring nodes until the DAG is restored. Finally, the network stabilizes into a new DAG, ready to route subsequent packets.

The decentralized nature of LRA offers several advantages. By enabling nodes to make routing decisions independently based on local information, LRA reduces communication overhead and enhances the protocol’s adaptability to dynamic conditions. In addition, proactive maintenance of a DAG minimizes latency, ensuring that routes are immediately available for data transmission.

However, LRA is not without limitations. Frequent topology changes can increase convergence time, delaying route stabilization. 

Despite these challenges, LRA and derivative algorithms remain effective solutions for scenarios requiring high adaptability and fault tolerance. The algorithm’s ability to dynamically adjust to changing network conditions makes it particularly valuable for applications in disaster recovery, military communications, and vehicular networks, where reliable connectivity is crucial.
\end{document}