\documentclass[../report.tex]{subfiles}
\graphicspath{{\subfix{../images/}}}

\begin{document}
\section{Network Simulator 3}
\textbf{Network Simulator 3} (\textbf{NS3}) is a discrete-event network simulator widely used for research and educational purposes in the field of networking. Designed as an open-source project, NS3 provides a modular and flexible framework for simulating and analyzing a wide variety of networking scenarios, ranging from wired and wireless networks to internet protocols and routing algorithms.

NS3 is built in C++ and provides Python bindings, making it accessible to a broad user base. It is specifically designed to facilitate realistic simulations by incorporating detailed models of network devices, protocols, and mobility patterns. NS3 supports various features, including the simulation of IP-based networks, wireless communication technologies such as Wi-Fi, LTE, and 5G, and many routing protocols like AODV, DSDV, and custom implementations.

NS3 is widely employed in academia and industry for protocol design and testing, performance analysis of existing protocols, and educational purposes. It enables optimization studies by providing detailed metrics for performance evaluation, including throughput, latency, and packet loss. In this study, NS3 was chosen as the simulation platform for implementing and evaluating the LRA routing protocol. Its flexibility and scalability enabled the modeling of a dynamic network environment with +100 nodes, random mobility, and varying traffic patterns. The detailed libraries and customizable features of NS3 facilitated the integration of the LRA protocol, ensuring realistic simulations and accurate performance metrics.

\end{document}